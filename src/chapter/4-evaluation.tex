\chapter{Evaluation}\label{ch:evaluation}

Die in Kapitel~\ref{ch:implementation} entstandene Software wurde im Verlauf dieser Arbeit einer umfangreichen Evaluierungsphase unterzogen.
Dadurch konnten individuelle Abläufe ermöglicht, unvorhergesehene Benutzerinteraktionen erfasst und ergonomische Bedienbarkeit sichergestellt werden.
In diesem Kapitel werden drei Anwendungsfälle betrachtet, in denen die Werkzeuge fulib.org und fulibFeedback verwendet wurden.
Dabei handelt es sich um drei Veranstaltungen der Universität Kassel aus dem Sommer- bis Wintersemester 2021 bis 2022.
In Abschnitt~\ref{sec:pm-2021-2022} wird zunächst die Veranstaltung "Programmieren und Modellieren"\footnote{
    Fachgebiet Softwaretechnik, Prof.\ Dr.\ Albert Zündorf.
} des Wintersemesters 2021/22 betrachtet.
Die Abschnitte~\ref{sec:algods-2021} und~\ref{sec:einfinf-2021-2022} bieten daraufhin Einblicke in die Veranstaltungen "Algorithmen und Datenstrukturen"\footnote{
    Fachgebiet Programmiersprachen/-Methodik, Prof.\ Dr.\ Claudia Fohry.\label{fn:fg-plm}
} im Sommersemester 2021 und "Einführung in die Informatik"\footref{fn:fg-plm} im Wintersemester 2021/22.

\todo{
    Referenzen:
    Eingewöhnungszeit~\ref{subsec:grading}.
    Statistiken~\ref{subsec:statistics}: Relative Anteile Code Search/Manuell, Code Search Savings, Punktzahl vs Bewertungsanzahl.
    fulibFeedback: Wahl von Codeabschnitten~\ref{subsec:choosing-code-snippets}.
}

\section{Programmieren und Modellieren}\label{sec:pm-2021-2022}

\todo{
    Erfahrungsbericht, User Feedback.
}

\section{Algorithmen und Datenstrukturen}\label{sec:algods-2021}

\todo{
    Keine Bewerter, nur Betrachten der Daten.
    Statistiken.
}

\section{Einführung in die Informatik}\label{sec:einfinf-2021-2022}

\todo{
    Erfahrungsbericht, User Feedback.
}
