\chapter{Ausblick}\label{ch:future-work}

Eine Vielzahl von Erweiterungen und neuen Komponenten wurde für diese Arbeit zunächst in Betracht gezogen, aber später verworfen.
Die Ideen, deren Nutzen und die Gründe, sie noch nicht umzusetzen, werden in diesem Kapitel diskutiert.
Dabei wird zwischen drei großen Gebieten unterschieden:
Integration weiterer Lernsysteme, Portierung von fulibFeedback auf andere \acp{ide} und Verbesserungen von Code Search.

\section{Integration weiterer Lernsysteme}

\todo{
    zB Moodle.
    Automatischer Import und Feedback
}

\section{fulibFeedback für andere \acsp{ide}}\label{sec:other-ides}

\todo{
    LSP erlaubt es.
    zB IntelliJ.
    fulib.org Projects für einfache Anwendungen ohne GUI.
}

\section{Verbesserungen von Code Search}\label{sec:code-search-improvements}

\todo{
    Platzhalter in Code Search Snippets.
    Umbenennung von Bezeichnern und andere Refactorings.
}

Umbenennung von Variablen bei in sich geschlossenen Codeschnipseln möglich.
Beispiel:

\begin{minted}{java}
public static void main(String[] args) {
    int i = 6;
    int j = 7;
    int answer = i * j;
    String greeting = "Hello";
    System.out.println(greeting);
    System.out.println("Answer: " + answer);
}
\end{minted}

Ersetzung von langen Bezeichnern durch Typ-Anfangsbuchstabe + laufende Nummer.
Beibehaltung von Offsets.

\begin{minted}{java}
public static void main(String[] s0  ) {
    int i = 6;
    int j = 7;
    int i0     = i * j;
    String s1       = "Hello";
    System.out.println(s1      );
    System.out.println("Answer: " + i0    );
}
\end{minted}

Problem: Variablendeklaration nicht zwangsweise in der Suchanfrage:

\begin{minted}{java}
    System.out.println(greeting);
    System.out.println("Answer: " + answer);
\end{minted}

greeting und answer können von außerhalb stammen;
Typ ist nicht bekannt -
Umbenennung kann nicht reproduziert werden.

Typ-Problem kann gelöst werden durch Benennung wie var0, var1, \ldots.
Nummerierung dann weiterhin Problem.

Alternative: Alle Bezeichner gleich behandeln - dann aber Verlust von Semantikinformationen.
