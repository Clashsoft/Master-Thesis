\chapter{Evaluation}\label{ch:evaluation}

Die in Kapitel~\ref{ch:implementation} entstandene Software wurde im Verlauf dieser Arbeit einer umfangreichen Evaluierungsphase unterzogen.
Dadurch konnten individuelle Abläufe ermöglicht, unvorhergesehene Benutzerinteraktionen erfasst und ergonomische Bedienbarkeit sichergestellt werden.
In diesem Kapitel werden drei Anwendungsfälle betrachtet, in denen die Werkzeuge fulib.org und fulibFeedback verwendet wurden.
Dabei handelt es sich um drei Veranstaltungen der Universität Kassel aus dem Sommer- bis Wintersemester 2021 bis 2022.
In Abschnitt~\ref{sec:pm-2021-2022} wird zunächst die Veranstaltung \ac{pm}\footnote{
    Fachgebiet Softwaretechnik, Prof.\ Dr.\ Albert Zündorf.
} des Wintersemesters 2021/22 betrachtet.
Die Abschnitte~\ref{sec:algods-2021} und~\ref{sec:einfinf-2021-2022} bieten daraufhin Einblicke in die Veranstaltungen "Algorithmen und Datenstrukturen"\footnote{
    Fachgebiet Programmiersprachen/-Methodik, Prof.\ Dr.\ Claudia Fohry.\label{fn:fg-plm}
} im Sommersemester 2021 und "Einführung in die Informatik"\footref{fn:fg-plm} im Wintersemester 2021/22.

\todo{
    Referenzen:
    Eingewöhnungszeit~\ref{subsec:grading}.
    Statistiken~\ref{subsec:statistics}: Relative Anteile Code Search/Manuell, Code Search Savings, Punktzahl vs Bewertungsanzahl.
    fulibFeedback: Wahl von Codeabschnitten~\ref{subsec:choosing-code-snippets}.
}

\section{\acl{pm}}\label{sec:pm-2021-2022}

Die Veranstaltung \ac{pm} stellt die größten Teil der Evaluation dar.
Sie umfasste ingesamt \todo{zwölf} Aufgabenblättern, welche von bis zu 125 Studierenden bearbeitet und bis zu sechs Personen mit fulib.org und fulibFeedback bewertet wurden.
Dabei sind individuelle Anmerkungen sowie eine große Anzahl von Metriken in Form der Statistik entstanden.
In diesem Abschnitt werden einige Ansichten der Benutzer erläutert (Abschnitt~\ref{subsec:user-feedback}) und die Rohdaten der Statistik erfasst (Abschnitt~\ref{subsec:pm-statistics}).
Des Weiteren wird erläutert, welche Erfahrung mit den Werkzeugen zu Anpassungen der Aufgabenstellungen und Bewertungskriterien geführt haben, um den Ablauf der Bewertung zu optimieren (Abschnitt~\ref{subsec:pm-adaptations}).

\subsection{Benutzer-Anmerkungen}\label{subsec:user-feedback}

\todo{
    Highlights aus \url{https://github.com/fujaba/fulib.org/issues/196}
}

\subsection{Statistiken}\label{subsec:pm-statistics}

\todo{
    Vergleich aller Hausaufgaben.
    Wichtigste Metriken: Anteil Code Search, Zeit bewertet, Zeit gespart.
    Besonderheiten bei Tasks.
}

\subsection{Anpassungen}\label{subsec:pm-adaptations}

\todo{
    Vorgaben für Namen, fx:ids.
    Beobachtungen bei Aufgabenstellung allgemein (Beispiel Todos -> mehr identische Strukturen, bessere Code Search Ergebnisse).
}

\todo{
    Erfahrungsbericht, User Feedback.
}

\section{Algorithmen und Datenstrukturen}\label{sec:algods-2021}

\todo{
    Keine Bewerter, nur Betrachten der Daten.
    Statistiken.
}

\section{Einführung in die Informatik}\label{sec:einfinf-2021-2022}

\todo{
    Erfahrungsbericht, User Feedback.
}
