\chapter{Einleitung}\label{ch:introduction}

Die Bewertung von Programmieraufgaben in einer Lehrveranstaltung mit vielen Teilnehmenden kann sehr aufwendig sein.
Abhängig von der Aufgabenstellung müssen viele Zeilen Quellcode gelesen und verstanden werden, um eine akkurate Einschätzung der Richtigkeit zu machen.
Die Arbeit der Bewertenden ist meist repetitiv und kann zu Flüchtigkeitsfehlern führen, wodurch Studierende möglicherweise nicht auf Fehler hingewiesen werden und einer Lerngelegenheit entgehen.
Dennoch bearbeitet jeder Studierende die gleichen Aufgaben.
Es sollte folglich naheliegen, dass auch ähnliche oder sogar gleiche Lösungen entstehen.
Diese Arbeit beabsichtigt, dieses Phänomen zu untersuchen und durch Automatisierung die Bewertenden zu entlasten.
Sie beschreibt Werkzeuge, die für diesen Anwendungszweck entwickelt wurden und bei der Bewertung gemäß der Ziele des nachfolgenden Abschnitts~\ref{sec:goals} unterstützen sollen.
Um einen breiteren Kontext zu schaffen, werden in Abschnitt~\ref{sec:related-work} verwandte Arbeiten betrachtet.
Die wissenschaftlichen Ziele werden in Abschnitt~\ref{sec:research-questions} anhand von \acp{rq} gesteckt.
Zuletzt gibt Abschnitt~\ref{sec:structure} eine kurze Übersicht über die Struktur dieser Arbeit.

\section{Zielsetzung}\label{sec:goals}

In dieser Arbeit soll Software entwickelt werden, die bei der Bewertung von Programmieraufgaben unterstützt.
Nachfolgend werden einige Aspekte beschrieben, die darüber hinaus angestrebt werden.

\begin{description}
    \item[Optionale Benutzung und Integration bestehender Vorgänge.]
    Die Werkzeuge sollen möglichst unaufdringlich sein, bestehende Ablaufe ergänzen und optional benutzbar sein.
    Insbesondere umfasst dies, dass Studierende sie nicht für die Abgabe verwenden müssen.
    Bewertungen sollen in einem Format eingereicht werden können, das auch ohne die Werkzeuge verfügbar ist.
    Beispielsweise soll die Verwendung von GitHub Classroom als Abgabemechanismus und GitHub Issues als Format für Feedback unterstützt werden.
    Anders ausgedrückt sollen die Werkzeuge für diese bestehenden Technologien eine Integration bieten.
    \item[Nachvollziehbarkeit.]
    Die Software soll die Konzepte Hausaufgabe, Teilaufgaben, Abgaben \ac{bzw} Lösungen, Bewertungen von Teilaufgaben trennen, sodass eine Gesamtbewertung ermöglicht wird.
    Insbesondere sollen Teilpunkte aufgezeichnet werden.
    \item[Übergreifende Bewertung.]
    Die gleichzeitige Bewertung mehrerer Lösungen soll erfolgen, indem Bewertungen von Teilaufgaben anhand von Codeabschnitten automatisch auf andere Lösungen mit ähnlichem Code übertragen werden.
    Dies soll in Echtzeit geschehen, damit andere Bewertende umgehend informiert werden und unnötige Arbeit vermieden wird.
    \item[Sicherheit.]
    Um Schäden durch böswilligen oder fahrlässigen Code zu vermeiden, soll das System keine Programme von Dritten ausführen.
    Stattdessen soll sämtliche automatische Bewertung anhand des Quelltexts durchgeführt werden.
\end{description}

\section{Verwandte Arbeiten}\label{sec:related-work}

\todo{
    Erste automatische Bewertung von Hollingsworth~\cite{hollingsworth-1960}.
    Es ist unmöglich, "ausführliche, gültige, verlässliche Bewertung" ohne Hilfswerkzeuge zu erstellen~\cite{kay-1994}.
    Vorteile nach~\cite{ala-mutka-2005}: Speed/Geschwindigkeit, Availability/Verfügbarkeit, Consistency/Konsistenz, Objectivity/Objektivität.
    Aufteilung in Teilaufgaben nach~\cite{enstroem-et-al-2011}.
    Codeassessor~\cite{vander-zanden-2012} gibt Programm vor, in dem Lücken ausgefüllt werden müssen.
    Existierende Tools für Plagiate MOSS~\cite{aiken-2002} und JPlag~\cite{prechelt-2003}.
    \cite{pieterse-2013}.
    ASSYST und Web-CAT.
}

\section{Forschungsfragen}\label{sec:research-questions}

Neben der Implementierung von Software sollen in dieser Arbeit auch einige Grundlagen im Gebiet der automatischen Bewertung erforscht werden.
Die folgenden vier Forschungsfragen sollen als Kern der Wissensschöpfung dienen.

\subsection[\acs{rq}1]{\ac{rq}1: Kann die Bewertung von Programmieraufgaben sinnvoll automatisiert werden?}\label{subsec:rq1-useful-automation}

Wie eingangs erwähnt, bietet sich die Automatisierung bei vielen Studierenden mit ähnlichen Lösungen an.
Dies setzt voraus, dass eine gemeinsame Aufgabenstellung vorhanden ist, wie es bei Hausaufgaben in Lehrveranstaltungen, nicht aber bei Projekten der Fall ist.
Mit dieser Frage soll erforscht werden, ob und welche Arten von Aufgaben besonders für die automatische Bewertung geeignet sind.

\subsection[\acs{rq}2]{\ac{rq}2: Welcher Mehraufwand wird für die automatisierte Bewertung benötigt?}\label{subsec:rq2-additional-effort}

Gemäß der Zielsetzung ist es naheliegend, dass bei einem optionalen Werkzeug grundsätzlich kein Mehraufwand zwingend erforderlich ist.
Um jedoch die möglichen Vorteile zu nutzen, müssen möglicherweise Schulung, Einrichtung und weitere Schritte bei der Bewertung erfolgen.
Weiterhin könnte dies umfassen, dass für die Programmieraufgaben Tests geschrieben oder Musterlösungen bereitgestellt werden.
Diese Frage soll untersuchen, ob diese weiteren Aufwände erheblich sind und wie sie in Relation zum Nutzen der Werkzeuge stehen.

\subsection[\acs{rq}3]{\ac{rq}3: Welche Effektivität und Effizienz kann von der Automatisierung erwartet werden?}\label{subsec:rq3-effectivity-efficiency}

In dieser Arbeit beschreibt der Begriff Effektivität den Anteil der Bewertungen von Teilaufgaben, die automatisch gemacht wurden, \ac{dh} in Relation zur Anzahl der händischen Bewertungen.
Je höher dieser prozentuale Wert, desto weniger Arbeit ist seitens der Bewertenden notwendig gewesen.
Die Effizienz hingegen bezeichnet die Zeit, die dabei eingespart wurde, genauer das Verhältnis von händischer Bewertungszeit und durch Automatisierung gesparter Arbeitszeit.
Es ist voraussehbar, dass diese Werte abhängig von der Aufgabenstellung und anderen Faktoren unterschiedlich ausfallen können.
Daher soll ergründet werden, welche Aufgaben positiven oder negativen Einfluss nehmen.
Zudem soll beobachtet werden, ob Unterschiede entstehen, wenn Lehrveranstaltungen aus niedrigen oder höheren Semestern oder Hausaufgaben vom Anfang statt Ende eines Vorlesungszeitraums betrachtet werden.

\subsection[\acs{rq}4]{\ac{rq}4: Wie können Aufgaben formuliert oder angepasst werden, um die Effektivität und Effizienz zu steigern?}\label{subsec:rq4-improve-effectivity-efficiency}

Aufbauend auf \acs{rq}3 ist naheliegend wünschenswert, die Effektivität oder Effizienz zu steigern, um Bewertungsarbeit zu reduzieren.
Diese Frage soll erforschen, wie dies durch Anpassung der Aufgabenstellung bezweckt werden kann.
Dabei sollen Anpassungen bestenfalls nicht oder nur geringfügig den Lehrinhalt verfälschen oder Mehraufwand erzeugen.

\section{Aufbau der Arbeit}\label{sec:structure}

Nachfolgend wird zunächst beschrieben, welche Konzepte dieser Arbeit zugrundeliegend.
Kapitel~\ref{ch:basics} enthält dafür eine genaue Beschreibung des Anwendungsfalls und der verwendeten Technologien, auf welche die Software aufbaut.
Daraufhin beschreibt~\ref{ch:implementation} im Detail die neuen Werkzeuge und deren Funktionsweise aus technischer Sicht.
Kapitel~\ref{ch:evaluation} wird dargelegt, wie die Software anhand von Benutzertests über einen längeren erprobt und iterativ verbessert wurde.
Ebenfalls werden dort erste Daten gesammelt, die für die Beantwortung der \acp{rq} notwendig sind.
In Kapitel~\ref{ch:results} werden diese schließlich ausgewertet und beantwortet.
Das Fazit in Kapitel~\ref{ch:conclusion} gibt einen abschließenden Rückblick auf die Erkenntnisse der Arbeit.
Zuletzt werden in Kapitel~\ref{ch:future-work} einige zukünftige Erweiterungsmöglichkeiten vorgestellt.
