\chapter{Fazit}\label{ch:conclusion}

Die Evaluation und Auswertung haben gezeigt, dass die in dieser Arbeit vorgestellten Werkzeuge, fulib.org und fulibFeedback, durchaus fähig sind, bei der Bewertung von Programmieraufgaben zu unterstützen.

% Code Search
Nach einer kurzen Einrichtung kann die vorgestellte Software die Produktivität steigern.
Indem Bewertungen von Teilaufgaben automatisch auf textuell ähnliche Lösungen übertragen werden, müssen Bewertende weniger Zeit investieren.
Dabei hat sich gezeigt, dass die reine Textsuche ohne Beachtung von Leerzeichen ausreicht, um eine große Anzahl von Ergebnissen mit nahezu identischem Code zu finden.
Die Wahl von Elasticsearch als Datenbank und Suchmaschine für Quellcode erwies sich als erfolgreich, da zufriedenstellendes Suchverhalten und Performance mit wenig Konfigurationsaufwand erzielt wurden.

% fulibFeedback
Nach anfänglichen Benutzungsproblemen konnte die \ac{vsc}-Erweiterung fulibFeedback die Abläufe auf der Weboberfläche fulib.org intuitiv ergänzen.
Die Entscheidung, beide Werkzeuge symbiotisch einzusetzen, anstatt die gesamte Bewertung mit fulibFeedback durchzuführen, nahm positiven Einfluss auf die allgemeine Benutzbarkeit der Software.
Das \acl{lsp} bewährte sich als modernes Rahmenwerk für die Implementierung von Editor-Funktionalität, das zukünftige Ausweitung auf andere \acp{ide} erlaubt.

% Evaluation
Die umfangreiche Evaluation deckte mit \ac{pm} einen vollständigen Vorlesungszeitraum ab und konnte anhand von \acl{algods} auf eine andere Veranstaltung ausgeweitet werden.
Dabei wurden wertvolle Daten gesammelt, welche die Leistungsfähigkeit und Zeitersparnis der Werkzeuge numerisch belegen.

% Vergleich mit anderen Vorgehen
Anders als bei gängigen Verfahren~\cite{jackson-1997-assyst,edwards-2008-web-cat,enstroem-et-al-2011,vander-zanden-2012}, welche die abgegebenen Programme ausführen und anhand von Verhaltensmustern bewerten, müssen für die Textsuche keine Musterlösungen, Testsuiten oder Rahmenvorgaben gemacht werden.
Studierende und Bewertende sind damit nicht in ihren Vorgehen und Abläufen eingeschränkt.
Obwohl keine vollständige Automatisierung erzielt werden kann, bietet diese Arbeit dennoch einen Ansatz, der bei mittleren bis großen Teilnehmerzahlen erhebliche Zeitersparnisse ermöglichen kann.
