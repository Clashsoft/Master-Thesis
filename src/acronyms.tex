\chapter*{Abkürzungsverzeichnis}

% Inhaltsverzeichnis und Kopfzeile
\addcontentsline{toc}{chapter}{Abkürzungsverzeichnis}
\markboth{Abkürzungsverzeichnis}{Abkürzungsverzeichnis}

\acrodefplural{ha}[HAs]{Hausaufgaben}

\begin{acronym}[XXXXXX]
    \acro{api}[API]{Application Programming Interface}
    \acro{ast}[AST]{Abstract Syntax Tree}
    \acro{css}[CSS]{Cascading Style Sheet}
    \acro{cst}[CST]{Concrete Syntax Tree}
    \acro{emf}[EMF]{Eclipse Modeling Framework}
    \acro{fxml}[FXML]{JavaFX XML}
    \acro{gui}[GUI]{Graphical User Interface}
    \acro{ha}[HA]{Hausaufgabe}
    \acro{html}[HTML]{Hypertext Markup Language}
    \acro{http}[HTTP]{Hypertext Transfer Protocol}
    \acro{https}[HTTPS]{Hypertext Transfer Protocol Secure}
    \acro{id}[ID]{Identifier}
    \acro{ide}[IDE]{Integrated Development Environment}
    \acro{ip}[IP]{Internet Protocol}
    \acro{json}[JSON]{JavaScript Object Notation}
    \acro{jvm}[JVM]{Java Virtual Machine}
    \acro{lsp}[LSP]{Language Server Protocol}
    \acro{pm}[PM]{Programmieren und Modellieren}
    \acro{rest}[REST]{Representational State Transfer}
    \acro{ssh}[SSH]{Secure Shell}
    \acro{ui}[UI]{User Interface}
    \acro{url}[URL]{Uniform Resource Locator}
    \acro{vsc}[VSCode]{Visual Studio Code}
    \acro{xml}[XML]{Extensible Markup Language}

    \vspace{\parskip}

    \acro{bzw}[bzw.]{beziehungsweise}
    \acro{dh}[d.h.]{das heißt}
    \acro{etc}[etc.]{et cetera}
    \acro{idR}[i.d.R.]{in der Regel}
    \acro{ggf}[ggf.]{gegebenenfalls}
    \acro{oBdA}[o.B.d.A.]{ohne Beschränkung der Allgemeinheit}
    \acro{sic}[sic]{sic erat scriptum}
    \acro{usw}[usw.]{und so weiter}
    \acro{uU}[u.U.]{unter Umständen}
    \acro{vgl}[vgl.]{vergleiche}
    \acro{zB}[z.B.]{zum Beispiel}
\end{acronym}
