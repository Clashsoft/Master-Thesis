\chapter{Einleitung}\label{ch:introduction}

\todo{
    Worum gehts, was ist das Problem, was beabsichtigt diese Arbeit, wie ist sie aufgebaut.
}

\section{Motivation}\label{sec:motivation}

\todo{
    Vereinfachen der PM-Bewertung durch Automatisierung.
    Repetitive Arbeit, oft ähnlicher Code.
}

\section{Zielsetzung}\label{sec:goals}

\todo{
    Tool zur Unterstützung der Bewertung:
    Genaues Aufzeichnen von Teilbewertungen/Teilpunkten.
    Gleichzeitige Bewertung mehrerer Lösungen durch Suchemechanismus.
    Lediglich interne Anpassung, kein Tool für Studentenverwendung, keine Änderung des bisherigen Workflows:
    GitHub Classroom Abgabe, Bewertungstabelle, GitHub Issues, Bewertungsrichtlinien in Markdown.
    Opt-in Verwendung.
}

\section{Verwandte Arbeiten}\label{sec:related-work}

\todo{
    Yang Hu et al, Re-factoring based Program Repair applied to Programming Assignments;
    Osei-Owusu et al, Grading-Based Test Suite Augmentation;
}

\section{Forschungsfragen}\label{sec:research-questions}

\subsection[RQ1]{RQ1: Kann die Bewertung von Programmieraufgaben sinnvoll automatisiert werden?}\label{subsec:rq1-useful-automation}

\todo{
    Relativ große Anzahl von Studierenden, ähnliche Lösungen - Automatisierung bietet sich an.
}

\subsection[RQ2]{RQ2: Welcher Mehraufwand wird für die automatisierte Bewertung benötigt?}\label{subsec:rq2-additional-effort}

\todo{
    Tests schreiben, Musterlösung, keiner?
}

\subsection[RQ3]{RQ3: Welche Effektivität und Effizienz kann von der Automatisierung erwartet werden?}\label{subsec:rq3-effectivity-efficiency}

\todo{
    Abhängig von Semester, Modul und Kursfortschritt.
}

\subsection[RQ4]{RQ4: Wie können Aufgaben formuliert oder angepasst werden, um die Effektivität und Effizienz zu steigern?}\label{subsec:rq4-improve-effectivity-efficiency}

\todo{
    Unter Beibehaltung der ursprünglichen Lehrziele, möglichst durch triviale Änderungen der Anforderungen.
}
