\chapter{Grundlagen}\label{ch:basics}

\section{Übungskonzept}\label{sec:programming-assignments}

\todo{
    Genauer Ablauf des Lebenszyklus einer Hausaufgabe.
    Personas von Studierenden und Bewertern.
}

\subsection{Ablauf}\label{subsec:workflow}

\todo{
    Konzeption, Punkteberechnung, Bearbeitung, Bewertung, Feedback.
}

\subsection{Studierende}\label{subsec:students}

\todo{
    Kreative Lösungsansätze oder Nachprogrammieren der Vorlesung.
}

\subsection{Betreuer}\label{subsec:teaching-assistants}

\todo{
    Viele Bewertungen am Stück oder nach und nach.
    Akribische Fehlersuche oder flüchtiges Betrachten.
}

\section{Technologien}\label{sec:tech}

\subsection{fulib.org}\label{subsec:fulib.org}

\todo{
    Screenshots?
}

fulib.org ist eine Webanwendung, die bereits 2019 im Rahmen von~\cite{explain} konzipiert und durch~\cite{bachelor-thesis} erweitert wurde.
Sie besteht aus einigen Modulen, die nachfolgend kurz beschrieben werden.

\paragraph{Scenarios}
Das erste Modul der Anwendung ist ein interaktiver Editor für die Szenario-Sprache aus~\cite{explain}.
Diese besteht aus textuellen Beispielszenarien, die einer festen Grammatik folgen.
Ein Compiler übersetzt die Szenarien in Java-Code und generiert dabei Klassen für ein Datenmodell mit~\cite{fulib}.
In~\cite{bachelor-thesis} wurde erstmals eine Erweiterung der Sprache umgesetzt, die gezielt die Bewertung von Aufgaben ermöglichen sollte.
Dafür wurde spezielle Syntax zur Mustererkennung auf Objektstrukturen entwickelt, die besonders für Fälle geeignet war, in der die Benennung von Variablen, Attributen und Assoziationen nicht festgelegt war.
Nachfolgend wird die Szenario-Sprache und der Editor auf fulib.org nicht weiter betrachtet.
Es handelt sich jedoch um wichtige Hintergründe und Erkenntnisse, die in den Abschnitten~\ref{sec:expanding-fulib.org} und~\ref{par:renaming-and-refactoring} wieder aufgegriffen werden.

\paragraph{Docs}
Hier kann die Dokumentation für verwandte Werkzeuge aus dem fulib-Toolkit nachgelesen werden.
Diese wird direkt von den jeweiligen GitHub-Repositories bezogen, ist also nicht Teil der Webanwendung selber.
Insbesondere befindet sich hier das Benutzerhandbuch von fulibFeedback.

\paragraph{Projects}
Dieser Teil der Webanwendung ist eine Prototyp für eine Online-\ac{ide}.
Dort können Projekte erstellt werden, die im Gegensatz zum Szenario-Editor mehrere Dateien umfassen können.
Insbesondere können hier Gradle- und Java-Dateien betrachtet und bearbeitet werden.
Ein Dateibaum zeigt Ordner und darin befindliche Dateien jeglicher Art.
Mit einem Web-Terminal können beliebige Kommandozeilenbefehle ausgeführt werden, darunter auch die zum Bauen und Ausführen verwendeten Gradle-Befehle.
Die Projekte ermöglichen folglich eine eingeschränkte Form der Anwendungsentwicklung in einer Desktop-\ac{ide}.
Nachfolgend werden Projekte nicht weiter eingesetzt, sie bieten sich jedoch für zukünftige Erweiterungen in Abschnitt~\ref{par:future-projects} an.

\paragraph{Assignments}
Das ursprünglich in~\cite{bachelor-thesis} entwickelte Assignments-Modul von fulib.org ist nun erneut Zentrum der Implementierung.
In Abschnitt~\ref{sec:expanding-fulib.org} wird vermehrt beleuchtet, welche Änderungen vorgenommen wurden.
Nun soll zunächst beschrieben werden, wie der Stand des Moduls gegen Ende von~\cite{bachelor-thesis} war.
Vor dieser Arbeit waren die Assignments, eine eigene Bezeichnung für Hausaufgabenblätter, nur auf Aufgaben in der Szenario-Sprache fokussiert.
Mit einfachen Formularen konnten Titel, Beschreibung, Abgabefrist und einige Teilaufgaben definiert werden.
Jede Teilaufgabe bestand mindestens aus Kurzbeschriebung und Punktzahl
Optional konnte eine Teilaufgabe mit Verifizierungscode in der Szenario-Sprache, meist mit der Pattern Matching-Syntax aus~\cite{bachelor-thesis}, ausgestattet werden.
Studierende erhielten nach erfolgreicher Erstellung des Assignments einen Einladungslink, unter dem sie eine Lösung hochladen konnten.
Diese bestand aus einem zusammenhängenden Szenario-Text.
Nach dem Absenden wurde für jede Teilaufgabe der Verifizierungscodes zusammen mit der Lösung ausgeführt.
Bei erfolgreicher Ausführung wurden für die Teilaufgabe volle Punktzahl, im Fehlerfall null Punkte vergeben.
Somit konnte eine Gesamtpunktzahl errechnet werden.

\subsection{Visual Studio Code}\label{subsec:visual-studio-code}

\ac{vsc}\footnote{\url{https://code.visualstudio.com/}} ist ein erweiterbarer Code-Editor von Microsoft.
Neben den Grundfunktionen eines Texteditors hat \ac{vsc} Funktionen zur Entwicklerproduktivitäts wie Syntax-Highlighting und Autovervollständigung.
Durch vorinstallierte Erweiterungen kann Versionsverwaltung mit Git verwendet werden.
Verschiedene Spracherweiterungen ermöglichen bessere Autovervollständigung und Semantisches Highlighting abhängig von Typinformationen sowie die Darstellung von Syntax-Fehlern und Warnmeldung im Quellcode.
Darüber hinaus ist die Anwendung beliebig mit Erweiterungen anpassbar, die aus einem von Microsoft bereitgestellten Marketplace bezogen werden können\footnote{\url{https://marketplace.visualstudio.com/VSCode}}.
Die Schnittstellen für Erweiterungen sind ausführlich dokumentiert und Entwickler können mit geringem Aufwand eigene Erweiterungen publizieren.
Für Spracherweiterung wirbt \ac{vsc} besonders mit dem \ac{lsp} (siehe~\ref{subsec:language-server-protocol}).
In Abschnitt~\ref{sec:fulibFeedback} wird dies genutzt, um die fulibFeedback-Erweiterung zu veröffentichen.

\todom{Vielleicht sollte das wo anders hin?}
\ac{vsc} wurde als Grundlage für diese Arbeit aus zwei Gründen gewählt.
Einerseits ist die Entwicklung von Erweiterungen im Vergleich zu der \ac{ide} IntelliJ IDEA deutlich einfacher, da die Logik mit dem \ac{lsp} wiederverwendbar implementiert werden kann.
IntelliJ IDEA bietet dafür keine eigene Unterstützung, stattdessen müssen Plugins von Dritten
\footnote{\url{https://github.com/gtache/intellij-lsp}\label{fn:intellij-lsp}}
\footnote{\url{https://github.com/lsp4intellij/intellij-lsp-plugin}\label{fn:intellij-lsp-plugin}}
\footnote{\url{https://github.com/ballerina-platform/lsp4intellij}\label{fn:lsp4intellij}}
verwendet werden.
Diese werden teilweise nicht weiterentwickelt\footref{fn:intellij-lsp} \footref{fn:intellij-lsp-plugin}, bieten nur eingeschränkte Funktionen des \ac{lsp}\footnote{\todo{Welche?}}, oder sind nicht mit aktuellen Versionen von IntelliJ IDEA kompatibel\footnote{\todo{Welche?}}.
Andererseits war \ac{vsc} ein vorgegebenes Werkzeug in der Veranstaltung "Programmieren und Modellieren" im Wintersemester 2021/22.
Es bot sich an, für die Bewertung die gleiche \ac{ide} zu benutzen wie die Studierenden zur Lösungserstellung.
Dadurch konnten bestimmte umgebungsabhängige Fehler vermieden werden.

\subsection{Language Server Protocol}\label{subsec:language-server-protocol}

Das \ac{lsp} bezeichnet eine von Microsoft entwickelte Spezifikation\footnote{\url{https://microsoft.github.io/language-server-protocol/}}, die ein Client/Server-Modell für Sprachunterstützung von Code-Editoren und ein zugehöriges Protokoll vorschlägt.
Das standardisierte Protokoll sollte einige Probleme lösen, die sowohl Entwicklern von Editoren als auch von Programmiersprachen bekannt waren.
Diese werden nachfolgend kurz erläutert.

\paragraph{Quadratischer Entwicklungsaufwand}
Im Vordergrund stand das Problem des quadratischen Entwicklungsaufwands ohne ein standardisiertes Protokoll.
Soll eine neue Programmiersprache Verwendung finden, ist es notwendig, Unterstützung in möglichst vielen Editoren zu implementieren.
Dies ist teilweise durch Plugins möglich, die von den Sprachautoren bereitgestellt werden können, aber mitunter signifikanten Entwicklungsaufwand benötigen.
Aus Sicht der Autoren hat jeder Editor andere Schnittstellen und Funktionen, die studiert und angebunden werden müssen.
Dies kann die Adaption von Sprachen bei begrenztem Entwicklungsbudget einschränken.
Aus Sicht der Editorautoren ergibt sich ein ähnliches Problem.
Um einen neuen Editor marktfähig zu machen, sollte dieser eine große Anzahl populärer Programmiersprachen unterstützen.
Diese können jedoch weitgehend unterschiedliche Schnittstellen und Werkzeuge bereitstellen, abhängig davon, ob die Editorunterstützung von den Sprachautoren bei der Konzeption von Compiler und anderen Tools eingearbeitet wurde.
Unter Umständen ist es notwendig, große Teile der Syntax und Semantik dieser Sprachen neu zu implementieren, um Editorunterstützung zu ermöglichen.
Insgesamt ergibt dies bei $n$ Editoren und $m$ Sprachen einen Aufwand von $n \cdot m$ notwendigen Integrationen.
Mit dem \ac{lsp} wird dieses Problem aus Sicht beider Seiten gelöst.
Sprachautoren können einen Server bereitstellen, der gegen die Schnittstellen des \ac{lsp} entwickelt wird.
Der Editor kann beliebige Sprachserver über die gleichen Schnittstellen ansprechen.
Folglich müssen nur $n + m$ Werkzeuge entwickelt werden.\cite{why-lsp}

\paragraph{Trennung von Technologien}
Ein weiterer Vorteil des \acp{lsp} ist die Möglichkeit, Server und Client in unterschiedlichen Programmiersprachen und Frameworks zu entwickeln.
Dies kann sprachseitig die Serverentwicklung vereinfachen, da beispielsweise Teile der Implementierung des Compilers wiederverwendet werden können.\cite{why-lsp}

\paragraph{Trennung von Prozessen}
Zuletzt nennt Microsoft die Prozesstrennung als vorteilhaft, welche die parallele Ausführung von rechenintensiven Aufgaben erlaubt.\cite{why-lsp}
Abhängig von der Architektur des Editors ist dies jedoch auch ohne ein solches Protokoll möglich.
\footnote{Beispielsweise in IntelliJ, das Vorgaben für Multithreading macht um schreibende Aktionen (Texteingabe, \ldots) von lesenden Aufgaben (Syntaxanalyse, Diagnostics, Highlighting, \ldots) zu trennen.\footnote{\url{https://plugins.jetbrains.com/docs/intellij/general-threading-rules.html}}}

Nachfolgend werden einige Editor-Funktionen beschrieben, die das \ac{lsp} anbietet.
Grundsätzlich sind weder Client noch Server von der Spezifikation verpflichtet, diese anzubieten.
Wird eine Editorfunktion von dem Server nicht unterstützt, so wird standardmäßig keine Aktion durchgeführt.
\footnote{\url{https://microsoft.github.io/language-server-protocol/overviews/lsp/overview/}}
Gleichermaßen kann der Server verschiedene Editoraktionen aufrufen, die nicht zwangsweise unterstützt werden müssen.
Beim Start des Servers wird aus diesem Grund kommuniziert, welche Funktionen beide Parteien bereitstellen.
\footnote{\url{https://microsoft.github.io/language-server-protocol/specifications/specification-3-17/\#initialize}}

\begin{itemize}
    \item \textbf{Autovervollständigung}.
    Dieses Feature kommt beim Schreiben von Quellcode zum Einsatz und soll die Produktivität von Entwicklern steigern.
    Eine einfache Form der Autovervollständigung kann trivial in einem Editor implementiert werden, indem die bereits im der aktuellen Datei verwendeten Wörter vorgeschlagen werden.
    In vielen Fällen ist dies jedoch nicht hilfreich, beispielsweise wenn die Syntax des Programms an der Stelle des Cursors bestimmte Arten von Bezeichnern verlangt, oder die Vorschläge abhängig von Typen sein sollen.
    Language Server können daher mit der Cursorposition nach Vorschlägen gefragt werden.
    \item \textbf{Zur Definition springen} und \textbf{Hover-Dokumentation}.
    Beim Lesen von Code ist es unter Umständen hilfreich, sich die Definition einer Klasse, Methode oder Variable ansehen zu können.
    Damit deren Position und Ursprungsdatei gefunden werden können, ist sprachabhängige Analyse notwendig, die ein Language Server bereitstellen kann.
    Ist lediglich die Dokumentation der Definition gefragt, kann diese beim Hovern über den Bezeichner angezeigt werden, falls dies implementiert wurde.
    \item \textbf{Diagnostics}.
    Meist handelt es sich hierbei um ein passives Feature, das nicht direkt vom Benutzer ausgelöst wird.
    Es ist hilfreich, während des Schreibens von Code mögliche Syntaxfehler oder andere Probleme direkt rot oder gelb unterstrichen sichtbar zu machen.
    Ein Language Server kann dies mittels Diagnostics implementieren.
    Im Gegensatz zu anderen Funktionen werden diese asynchron vom Client/Editor angefragt.
    Der Server kann dann Diagnostics ermitteln und das Ergebnis nach einiger Zeit per Push an den Client senden, um die Meldungen anzuzeigen.
    \item \textbf{Code Actions} und Refactorings.
    Ein Language Server kann anhand der aktuellen Cursorposition oder Auswahl eine oder mehrere Aktionen bereitstellen.
    Diese werden im Editor in einem Kontektmenü oder ähnlichem angezeigt.
    Eine Aktion kann, wenn sie vom Benutzer ausgewählt wird, Änderungen am Text oder anderen Dateien durchführen.
    Beispielsweise können dadurch einfache Refactorings wie das Schachteln in einer neuen Schleife implementiert werden.
    Nicht möglich ist das Erfragen von weiterem Input des Benutzers.
    Dadurch können mit Code Actions keine komplexeren Refactorings wie Umbenennen oder Methode Extrahieren umgesetzt werden.
    Ersteres hat aus diesem Grund eine eigene Schnittstelle.
\end{itemize}

Trotz der Bezeichnung \textbf{Language} Server Protocol ist dieses nicht nur für den Einsatz für Programmiersprachen geeignet.
Mit den bereitgestellten Schnittstellen können auch andere Entwicklerwerkzeuge implementiert werden.
Durch Diagnostics können beispielsweise auch Rechtschreibprüfung oder Linter\footnote{Programme, die Quellcode anhand von verschiedener Analyseverfahren auf häufige Fehlerquellen untersuchen} umgesetzt werden.
In Abschnitt~\ref{sec:fulibFeedback} wird erläutert, wie die fulibFeedback-Erweiterung die Diagnostics und Code Actions des \ac{lsp} anwendet.

\subsection{Elasticsearch}\label{subsec:elasticsearch}

Elasticsearch\footnote{\url{https://www.elastic.co/elasticsearch/}} bezeichnet eine Suchmaschine und dokumentorientierte Datenbank\footnote{Im Gegensatz zu tabellenbasierten Datenbanken erlauben Dokumente meist schemalose und geschachtelte Daten.}, die für verschiedene textuelle und strukturelle Anfragen optimiert ist.
Neben der Textsuche bietet Elasticsearch auch Lösungen für verwandte Probleme wie Log-Analyse, Metriken, Datentrends, Geo-Anfragen und Anwendungen der Genetik (Bioinformatik).
Diese sind jedoch im Folgenden nicht relevant und werden daher nicht weiter erläutert.

Die Textsuche basiert in großen Teilen auf der quelloffenenen Java-Bibliothek Lucene\footnote{\url{https://lucene.apache.org/}} der Apache Foundation.
Elasticsearch implementiert Teile der Textanalyse, Indexierung, Suche, und Highlighting mit Lucene.
Diese werden nachfolgend datailliert beschrieben.

\paragraph{Analyse}
Die Analyse von Textdaten umfasst im Fall von natürlicher Sprache zwei Schritte, die Tokenisierung und Normalisierung.
Bei der Tokenisierung wird ein zusammenhängender Text in sogenannte Tokens aufgeteilt, welche meist einzelne Wörter abbilden.
Dabei werden standardmäßig Leerzeichen und Zeichen wie Punkte, Kommata oder Anführungsstriche verworfen.
Die Normalisierung ändert die entstandenen Tokens, sodass bei der Suche ähnliche Wörter gleich behandelt werden können.
Dafür werden beispielsweise Groß- und Kleinschreibung verworfen, Plural zu Singular geändert, der Wortstamm gebildet (Stemming), oder Synonyme verwendet.
Weiterhin können sogenannte Stepwords, Wörter die keine semantische Relevanz im Text haben, in der Normalisierung gefiltert werden.
Um dies sinnvoll umsetzen zu können, muss für gewöhnlich die zugrundeliegende Sprache des Textes definiert werden.
Tokenisierung und Normalisierung können dann sprachspezifische Syntaxregeln und Wörterbücher verwenden.
\footnote{\url{https://www.elastic.co/guide/en/elasticsearch/reference/current/analysis-overview.html}}
Es ist möglich, eigene Analyseverfahren zu definieren, die aus Zeichenfiltern, Tokenisierer und Stepwordfiltern bestehen.
\footnote{\url{https://www.elastic.co/guide/en/elasticsearch/reference/current/analysis-custom-analyzer.html}}
So kann beispielsweise die Textanalyse für Programmiersprachen definiert werden, wie in Abschnitt~\ref{subsec:solution-detail-and-evaluation} demonstriert wird.

\paragraph{Indexierung}
Ein Index dient der Suchoptimierung, um häufige Anfragen schneller duchführen zu können.
Die genaue Form und verwendete Datenstruktur hängt davon ab, welcher Datentyp verwendet wird und für welche Art von Suchanfragen der Index konfiguriert wurde.
Ein einfacher Index kann beispielsweise eine Menge von Tokens speichern, die im Text vorkommen.
Dabei werden die Häufigkeit, Position und Form der Wörter im Originaltext nicht beachtet.
Wird eine Suchanfrage mit gleicher Tokenisierung und Normalisierung analysiert, entsteht eine vergleichbare Menge.
Ist die aus der Suchanfrage entstehende Menge eine Teilmenge der Tokens des Originaltexts, ist das Originaldokument ein gültiges Suchergebnis.
Dafür ist für jedes Dokument lediglich eine Mengenoperation notwendig und kein Auslesen und Vergleichen jedes einzelnen Zeichens des Originaltexts.
\footnote{\url{https://www.elastic.co/guide/en/elasticsearch/reference/current/analysis-index-search-time.html}}

\paragraph{Suche}
Für die Textsuche bietet Elasticsearch verschiedenen Arten von Anfragen und Optionen.
Einfache Beispiele für Anfragetypen sind die exakte Übereinstimmung, Textsuche, Wildcards und die Suche mit regulären Ausdrücken.
\footnote{\url{https://www.elastic.co/guide/en/elasticsearch/reference/current/query-dsl.html}}
Diese können mit bool'schen Operatoren kombiniert werden, wobei zwischen Bedingungen, die das Ranking von Ergebnissen beeinflussen, und Filtern, welche die Inklusion von Ergebnissen ohne Einfluss auf das Ranking kontrollieren, unterschieden wird.
\footnote{\url{https://www.elastic.co/guide/en/elasticsearch/reference/current/query-dsl-bool-query.html}}
Ferner können Optionen für Sortierung, Aggregationen und generelle Datenkontrolle (Anzahl der Ergebnisse, Pagination\footnote{Aufteilung von einer großen Anzahl von Suchergebnissen (>1000) auf mehrere Seiten.}, Timeout, Asynchronität, etc.) angegeben werden.
\footnote{\url{https://www.elastic.co/guide/en/elasticsearch/reference/current/search-your-data.html}}

\paragraph{Highlighting}
Gewöhnliche Suchanfragen geben lediglich die Ergebnisse zurück, die mit der Anfrage übereinstimmen, jedoch nicht, an welcher Stelle im Text die gesuchten Wörter vorkommen.
Mit einem Highlighter, der als Suchoption aus Performancegründen aktiviert werden muss, können diese Stellen gefunden werden, um beispielsweise auf einer Seite mit Suchergebnissen die Wörter farblich hervorzuheben.
\footnote{\url{https://www.elastic.co/guide/en/elasticsearch/reference/current/highlighting.html}}
In Abschnitt~\ref{subsec:solution-detail-and-evaluation} wird dies eingesetzt, um den Originalquellcode nach einer Suchanfrage zu rekonstruieren.

\subsection{Sonstige}\label{subsec:other-libraries}

\todo{
    Angular, NestJS, Bootstrap, Bootstrap-Darkmode, Bibliotheken.
}
