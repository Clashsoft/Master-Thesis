\chapter{Grundlagen}\label{ch:basics}

\section{Akademische Programmieraufgaben}\label{sec:programming-assignments}

\todo{
    Genauer Ablauf des Lebenszyklus einer Hausaufgabe.
    Personas von Studierenden und Bewertern.
}

\subsection{Ablauf}\label{subsec:workflow}

\todo{
    Konzeption, Punkteberechnung, Bearbeitung, Bewertung, Feedback.
}

\subsection{Studierende}\label{subsec:students}

\todo{
    Kreative Lösungsansätze oder Nachprogrammieren der Vorlesung.
}

\subsection{Betreuer}\label{subsec:teaching-assistants}

\todo{
    Viele Bewertungen am Stück oder nach und nach.
    Akribische Fehlersuche oder flüchtiges Betrachten.
}

\section{Technologien}\label{sec:tech}

\subsection{fulib.org}\label{subsec:fulib.org}

\todo{
    Platform und Grundlage für die Implementierung.
    Erweiterung der Assignments aus~\cite{bachelor-thesis}.
}

\subsection{Visual Studio Code}\label{subsec:visual-studio-code}

\todo{
    Umgebung für fulibFeedback plugin.
    Gewählt wegen modernen Plugin API (LSP), Anwendung in der Fallstudie, Einfachheit gegenüber IntelliJ.
}

\subsection{Language Server Protocol}\label{subsec:language-server}

\todo{
    Wiederverwendbares Framework, theoretisch einfach Anbindung in andere IDEs.
    Einfache Implementierung von Diagnostics und Code Actions (später Selection).
}

\subsection{Elasticsearch}\label{subsec:elasticsearch}

\todo{
    Datenbank für Textsuche optimiert.
    Basiert auf Apache Lucene.
    Eigentlich für natürliche Sprache ausgelegt (N-Gramme, Stemming).
    Erweiterbar mit Custom Lexer.
}

\subsection{Sonstige}\label{subsec:other-libraries}

\todo{
    Angular, NestJS, Bootstrap, Bootstrap-Darkmode, Bibliotheken.
}
