\chapter{Ausblick}\label{ch:future-work}

Eine Vielzahl von Erweiterungen und neuen Komponenten wurde für diese Arbeit zunächst in Betracht gezogen, aber später verworfen.
Die Ideen, deren Nutzen und die Gründe, sie noch nicht umzusetzen, werden in diesem Kapitel diskutiert.
Dabei wird zwischen drei großen Gebieten unterschieden:
Integration weiterer Lernsysteme, Portierung von fulibFeedback auf andere \acp{ide} und Verbesserungen von Code Search.

\section{Integration weiterer Lernsysteme}

Bisher unterstützt fulib.org den automatischen Import von Lösungen mit GitHub Classroom und das Hochladen von Archiven.
Dies deckt die Anwendungsfälle von den Veranstaltungen aus Kapitel~\ref{ch:evaluation} ab, doch verwenden nicht alle Lehrenden diese Plattform oder Formate.
Bei Abweichungen kann es zu Mehraufwand kommen, wenn beispielsweise erst Dateien in Archive verpackt werden müssen.
Folglich kann es sinnvoll sein, auch andere Lernsysteme zu integrieren.

Ein Beispiel für ein solches Lernsystem ist Moodle\footnote{\url{https://moodle.org/}}.
Dort können Lehrende ebenfalls Assignments erstellen\footnote{\url{https://docs.moodle.org/311/en/Assignment_activity}}, für die Studierenden Lösungen hochladen können.
Bewertende können die Abgaben einsehen und Punkte mit Feedback vergeben.
Die Moodle-\ac{api}\footnote{\url{https://docs.moodle.org/dev/Core_APIs}} bietet Schnittstellen um maschinell die Lösungen auszulesen und Bewertungen zu hinterlegen.

\section{fulibFeedback für andere \acsp{ide}}\label{sec:other-ides}

In Abschnitt~\ref{subsec:language-server-protocol} wurde bereits dargelegt, dass das \ac{lsp} für Wiederverwendbarkeit ausgelegt ist.
Es liegt daher nahe, dass die \ac{vsc}-Erweiterung fulibFeedback auch für andere \acp{ide} mit \ac{lsp}-Unterstützung portiert werden kann.
Konkret wurde in Betracht gezogen, dies für IntelliJ zu tun, da die Entwicklungsumgebung ebenfalls in den evaluierten Veranstaltungen eingesetzt wurde.
Das Vorhaben wurde jedoch verworfen, da in \ac{pm} die Vorgabe von \ac{vsc} bestand und diese \ac{ide} auch für die Bewertung verwendet wurde.
Darüber hinaus ist die Unterstützung des \ac{lsp} seitens IntelliJ nicht optimal, wie in Abschnitt~\ref{subsec:visual-studio-code} erwähnt wurde.

Eine weitere Möglichkeit ist das Projects-Modul von fulib.org, das in Abschnitt~\ref{subsec:fulib.org} beschrieben wurde.
Die vereinfachte Online-\ac{ide} könnte ebenfalls die Funktionen von fulibFeedback anbieten und dabei besser mit fulib.org integriert sein, als es mit einer Erweiterung möglich wäre.

\todo{
    fulib.org Projects für einfache Anwendungen ohne GUI.
}

\section{Verbesserungen von Code Search}\label{sec:code-search-improvements}

\todo{
    Platzhalter in Code Search Snippets.
    Umbenennung von Bezeichnern und andere Refactorings.
}

Umbenennung von Variablen bei in sich geschlossenen Codeschnipseln möglich.
Beispiel:

\begin{minted}{java}
public static void main(String[] args) {
    int i = 6;
    int j = 7;
    int answer = i * j;
    String greeting = "Hello";
    System.out.println(greeting);
    System.out.println("Answer: " + answer);
}
\end{minted}

Ersetzung von langen Bezeichnern durch Typ-Anfangsbuchstabe + laufende Nummer.
Beibehaltung von Offsets.

\begin{minted}{java}
public static void main(String[] s0  ) {
    int i = 6;
    int j = 7;
    int i0     = i * j;
    String s1       = "Hello";
    System.out.println(s1      );
    System.out.println("Answer: " + i0    );
}
\end{minted}

Problem: Variablendeklaration nicht zwangsweise in der Suchanfrage:

\begin{minted}{java}
    System.out.println(greeting);
    System.out.println("Answer: " + answer);
\end{minted}

greeting und answer können von außerhalb stammen;
Typ ist nicht bekannt -
Umbenennung kann nicht reproduziert werden.

Typ-Problem kann gelöst werden durch Benennung wie var0, var1, \ldots.
Nummerierung dann weiterhin Problem.

Alternative: Alle Bezeichner gleich behandeln - dann aber Verlust von Semantikinformationen.
