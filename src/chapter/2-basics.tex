\chapter{Grundlagen}\label{ch:basics}

\section{Übungskonzept}\label{sec:programming-assignments}

\todo{
    Genauer Ablauf des Lebenszyklus einer Hausaufgabe.
    Personas von Studierenden und Bewertern.
}

\subsection{Ablauf}\label{subsec:workflow}

\todo{
    Konzeption, Punkteberechnung, Bearbeitung, Bewertung, Feedback.
}

\subsection{Studierende}\label{subsec:students}

\todo{
    Kreative Lösungsansätze oder Nachprogrammieren der Vorlesung.
}

\subsection{Betreuer}\label{subsec:teaching-assistants}

\todo{
    Viele Bewertungen am Stück oder nach und nach.
    Akribische Fehlersuche oder flüchtiges Betrachten.
}

\section{Technologien}\label{sec:tech}

\subsection{fulib.org}\label{subsec:fulib.org}

\todo{
    Screenshots?
}

fulib.org ist eine Webanwendung, die bereits 2019 im Rahmen von~\cite{explain} konzipiert und durch~\cite{bachelor-thesis} erweitert wurde.
Sie besteht aus einigen Modulen, die nachfolgend kurz beschrieben werden.

\paragraph{Scenarios}
Das erste Modul der Anwendung ist ein interaktiver Editor für die Szenario-Sprache aus~\cite{explain}.
Diese besteht aus textuellen Beispielszenarien, die einer festen Grammatik folgen.
Ein Compiler übersetzt die Szenarien in Java-Code und generiert dabei Klassen für ein Datenmodell mit~\cite{fulib}.
In~\cite{bachelor-thesis} wurde erstmals eine Erweiterung der Sprache umgesetzt, die gezielt die Bewertung von Aufgaben ermöglichen sollte.
Dafür wurde spezielle Syntax zur Mustererkennung auf Objektstrukturen entwickelt, die besonders für Fälle geeignet war, in der die Benennung von Variablen, Attributen und Assoziationen nicht festgelegt war.
Nachfolgend wird die Szenario-Sprache und der Editor auf fulib.org nicht weiter betrachtet.
Es handelt sich jedoch um wichtige Hintergründe und Erkenntnisse, die in den Abschnitten~\ref{sec:expanding-fulib.org} und~\ref{par:renaming-and-refactoring} wieder aufgegriffen werden.

\paragraph{Docs}
Hier kann die Dokumentation für verwandte Werkzeuge aus dem fulib-Toolkit nachgelesen werden.
Diese wird direkt von den jeweiligen GitHub-Repositories bezogen, ist also nicht Teil der Webanwendung selber.
Insbesondere befindet sich hier das Benutzerhandbuch von fulibFeedback.

\paragraph{Projects}
Dieser Teil der Webanwendung ist eine Prototyp für eine Online-\ac{ide}.
Dort können Projekte erstellt werden, die im Gegensatz zum Szenario-Editor mehrere Dateien umfassen können.
Insbesondere können hier Gradle- und Java-Dateien betrachtet und bearbeitet werden.
Ein Dateibaum zeigt Ordner und darin befindliche Dateien jeglicher Art.
Mit einem Web-Terminal können beliebige Kommandozeilenbefehle ausgeführt werden, darunter auch die zum Bauen und Ausführen verwendeten Gradle-Befehle.
Die Projekte ermöglichen folglich eine eingeschränkte Form der Anwendungsentwicklung in einer Desktop-\ac{ide}.
Nachfolgend werden Projekte nicht weiter eingesetzt, sie bieten sich jedoch für zukünftige Erweiterungen in Abschnitt~\ref{par:future-projects} an.

\paragraph{Assignments}
Das ursprünglich in~\cite{bachelor-thesis} entwickelte Assignments-Modul von fulib.org ist nun erneut Zentrum der Implementierung.
In Abschnitt~\ref{sec:expanding-fulib.org} wird vermehrt beleuchtet, welche Änderungen vorgenommen wurden.
Nun soll zunächst beschrieben werden, wie der Stand des Moduls gegen Ende von~\cite{bachelor-thesis} war.
Vor dieser Arbeit waren die Assignments, eine eigene Bezeichnung für Hausaufgabenblätter, nur auf Aufgaben in der Szenario-Sprache fokussiert.
Mit einfachen Formularen konnten Titel, Beschreibung, Abgabefrist und einige Teilaufgaben definiert werden.
Jede Teilaufgabe bestand mindestens aus Kurzbeschriebung und Punktzahl
Optional konnte eine Teilaufgabe mit Verifizierungscode in der Szenario-Sprache, meist mit der Pattern Matching-Syntax aus~\cite{bachelor-thesis}, ausgestattet werden.
Studierende erhielten nach erfolgreicher Erstellung des Assignments einen Einladungslink, unter dem sie eine Lösung hochladen konnten.
Diese bestand aus einem zusammenhängenden Szenario-Text.
Nach dem Absenden wurde für jede Teilaufgabe der Verifizierungscodes zusammen mit der Lösung ausgeführt.
Bei erfolgreicher Ausführung wurden für die Teilaufgabe volle Punktzahl, im Fehlerfall null Punkte vergeben.
Somit konnte eine Gesamtpunktzahl errechnet werden.

\subsection{Visual Studio Code}\label{subsec:visual-studio-code}

\todo{
    Umgebung für fulibFeedback plugin.
    Gewählt wegen modernen Plugin API (LSP), Anwendung in der Fallstudie, Einfachheit gegenüber IntelliJ.
}

\subsection{Language Server Protocol}\label{subsec:language-server-protocol}

\todo{
    Wiederverwendbares Framework, theoretisch einfach Anbindung in andere IDEs.
    Einfache Implementierung von Diagnostics und Code Actions (später Selection).
}

\subsection{Elasticsearch}\label{subsec:elasticsearch}

\todo{
    Datenbank für Textsuche optimiert.
    Basiert auf Apache Lucene.
    Eigentlich für natürliche Sprache ausgelegt (N-Gramme, Stemming).
    Erweiterbar mit Custom Lexer.
}

\subsection{Sonstige}\label{subsec:other-libraries}

\todo{
    Angular, NestJS, Bootstrap, Bootstrap-Darkmode, Bibliotheken.
}
