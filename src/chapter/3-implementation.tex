\chapter{Implementierung}\label{ch:implementation}

\todo{
    Einleitungssatz.
}

In diesem Kapitel wird die Hausaufgabe 3\footnote{\url{https://seblog.cs.uni-kassel.de/wp-content/uploads/2021/11/PM2022_Hausaufgabe03.pdf}} aus der Veranstaltung Programmieren und Modellieren im Wintersemester 2021/22 an der Universität Kassel als laufendes Beispiel verwendet.
Die Lernziele der Hausaufgabe waren die Übersetzung eines Klassendiagramms in Java-Code, die damit verbundene Implementierung von Referenzieller Integrität\footnote{Dies bezeichnet ein Verhalten, bei dem durch Assoziationen verknüpfte Objekte stets in beide Richtungen konsistent verlinkt sind.}, sowie das korrekte Testen des dabei entstehenden Programmcodes.
Diese Hausaufgabe wurde gewählt, da sie sowohl individuellen als auch schematischen Code von Studierenden erwartet.
Zudem handelt es sich um eine Hausaufgabe aus der Anfangsphase der Veranstaltung, in der mit einer höheren Abgabenanzahl und -Vielfalt bei gleichzeitig geringerer Schwierigkeit und Komplexität im Vergleich zu späteren Aufgaben zu rechnen ist.

Die Implementierung dieser Arbeit besteht aus zwei weitgehend getrennten Projekten, die jedoch miteinander kommunizieren und integriert sind.
Abschnitt~\ref{sec:expanding-fulib.org} beschreibt zunächst die Änderung, die an der Webanwendung fulib.org durchgeführt wurden.
Das dabei entstandene Werkzeug ist bis auf wenige Ausnahmen autonom für die Bewertung von Abgaben einsetzbar.
Als Erweiterung oder zusätzliches Hilfsmittel dient die \ac{vsc}-Erweiterung fulibFeedback, die in Abschnitt~\ref{sec:fulibFeedback} erläutert wird.
Insbesondere kann diese Bewertende bei der Bewertung und Studierende bei der Berichtigung von Quellcode unterstützen.
Ohne fulibFeedback sind die Bewertungsmechanismen von fulib.org nur unabhängig von Quellcode nutzbar.

\section{Erweiterung von fulib.org}\label{sec:expanding-fulib.org}

Wesentlicher Teil der Implementierung ist die Erweiterung von fulib.org durch Hinzufügen neuer und Anpassung alter Funktionalität.
In Abschnitt~\ref{subsec:fulib.org} wurde bereits die Modulaufteilung und der Stand vor Beginn dieser Arbeit beschrieben.
Nachfolgend wird ein detaillierter Ablauf erläutert, der für die Bewertung von Hausaufgaben notwendig ist.
Dieser beginnt mit dem Erstellen von Assignments in Abschnitt~\ref{subsec:creating-assignments}.
Daraufhin werden in Abschnitt~\ref{subsec:grading} die Schritte beschrieben, die bei der Bewertung getätigt werden.
Abschnitt~\ref{subsec:statistics} zeigt, wie mithilfe der Statistiken eine Einsicht in die numerischen Hintergründe eines Assignments geboten wird.
Zuletzt wird die sogenannte Code Search-Technologie vorgestellt, die eine Suchmaschine für Quellcode in Abgaben bereitstellt.
Dies ist Inhalt von Abschnitt~\ref{subsec:code-search}.

\subsection{Erstellen von Assignments}\label{subsec:creating-assignments}

\todo{
    Teilaufgaben, Punkteberechnung.
    GitHub Classroom Import.
}

\subsection{Bewertung}\label{subsec:grading}

\todo{
    Sharing.
    QoL in Tabelle.
    Oberfläche für Evaluation und Snippets.
    Feedback mit GitHub Classroom, Email, etc.
}

\subsection{Statistiken}\label{subsec:statistics}

\todo{
    Evaluations als Darstellung der Code Search-Effektivität.
    Task-Liste als Information über schwierigste Teilaufgaben / Problemquellen.
}

\subsection{Code Search}\label{subsec:code-search}

\todo{
    Code Search als Anwendung von Elasticsearch.
}

\section{fulibFeedback}\label{sec:fulibFeedback}

\subsection{VSCode Extension}\label{subsec:vscode-extension}

\todo{
    Einfacher Client für Language Server.
    Einstellungen.
    Protocol Handler für Konfiguration.
}

\subsection{Language Server}\label{subsec:language-server}

\todo{
    Selection und Diagnostics.
    Wiederverwendbar für andere IDEs.
}
