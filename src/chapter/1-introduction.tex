\chapter{Einleitung}\label{ch:introduction}

Die Bewertung von Programmieraufgaben in einer Lehrveranstaltung mit vielen Teilnehmenden kann sehr aufwendig sein.
Abhängig von der Aufgabenstellung müssen viele Zeilen Quellcode gelesen und verstanden werden, um akkurat die Richtigkeit einzuschätzen.
Die Arbeit der Bewertenden ist meist repetitiv und kann zu Flüchtigkeitsfehlern führen, wodurch Studierende möglicherweise nicht auf Fehler hingewiesen werden und eine Lerngelegenheit versäumen.
Alle Studierenden bearbeiten jedoch die gleichen Aufgaben.
Folglich sollte es naheliegen, dass auch ähnliche Lösungen entstehen.
Die vorliegende Arbeit beabsichtigt, dieses Phänomen zu untersuchen und durch Automatisierung die Bewertenden zu entlasten.
Sie stellt Werkzeuge vor, die für diesen Anwendungszweck entwickelt wurden und bei der Bewertung gemäß der Zielsetzung (Abschnitt~\ref{sec:goals}) unterstützen sollen.
Um einen breiteren Kontext zu schaffen, werden in Abschnitt~\ref{sec:related-work} verwandte Arbeiten diskutiert.
Die wissenschaftlichen Ziele werden in Abschnitt~\ref{sec:research-questions} anhand von \acp{rq} gesetzt.
Zuletzt gibt Abschnitt~\ref{sec:structure} einen tieferen Einblick in die Struktur dieser Arbeit.

\section{Zielsetzung}\label{sec:goals}

Diese Arbeit verfolgt das allgemeine Ziel, Software-Werkzeuge zu entwickeln, die bei der Bewertung von Programmieraufgaben unterstützt.
Nachfolgend werden einige Aspekte beschrieben, die darüber hinaus angestrebt werden.

\begin{description}
    \item[Optionale Benutzung und Integration bestehender Vorgänge.]
    Die Software soll bestehende Abläufe ergänzen und optional anwendbar sein.
    Insbesondere umfasst dies, dass Studierende die Software nicht für die Abgabe verwenden müssen.
    Bewertungen sollen in einem Format eingereicht werden können, welches auch ohne die Werkzeuge verfügbar ist.
    Beispielsweise soll die Verwendung von GitHub Classroom als Abgabemechanismus und GitHub Issues als Format für Feedback unterstützt werden.
    In anderen Worten sollen die Werkzeuge eine Integration für diese bestehenden Technologien bieten.
    \item[Nachvollziehbarkeit.]
    Die Software soll die Konzepte Hausaufgaben, Teilaufgaben, Abgaben \ac{bzw} Lösungen und Bewertungen von Teilaufgaben trennen, sodass eine Gesamtbewertung ermöglicht wird.
    Sowohl Gesamtpunktzahl als auch deren Zusammensetzung aus einzelnen Teilpunkte sollen aufgezeichnet werden.
    \item[Übergreifende Bewertung.]
    Die übergreifende \ac{bzw} gleichzeitige Bewertung mehrerer Lösungen soll erfolgen, indem Bewertungen von Teilaufgaben anhand von Codeabschnitten automatisch auf andere Lösungen mit ähnlichem Code übertragen werden.
    Dies soll ohne spürbare Verzögerung geschehen, damit andere Bewertende umgehend informiert werden und keine vermeidbare Arbeit anfällt.
    \item[Sicherheit.]
    Um Schäden durch böswilligen oder fahrlässigen Code zu vermeiden, soll das System keine Programme von Dritten ausführen.
    Stattdessen soll sämtliche automatische Bewertung lesend anhand des Quelltexts durchgeführt werden.
\end{description}

\section{Verwandte Arbeiten}\label{sec:related-work}

Die automatische Bewertung von Programmieraufgaben wurde bereits in den Anfängen der Informatiklehre betrachtet~\cite{hollingsworth-1960}.
Besonders für große Online-Kurse mit bis zu vierstelligen Teilnehmerzahlen handelt es sich um ein Aufwandsproblem~\cite{pieterse-2013}.
Seitdem wurde eine Vielzahl von Werkzeugen entwickelt, um das Problem zu lösen~\cite{jackson-1997-assyst,edwards-2008-web-cat,enstroem-et-al-2011,vander-zanden-2012}.
Teilweise wurde dabei unvollständiger Quellcode vorgegeben, in dem Lücken ausgefüllt werden mussten~\cite{vander-zanden-2012}.
Mit der automatischen Bewertung wurden Vorteile wie "Geschwindigkeit, Verfügbarkeit, Konsistenz und Objektivität"~\cite{ala-mutka-2005}\footnote{
    Übersetzt aus dem Englischen: "speed, availability, consistency and objectivity".
}, ferner sogar die "Unmöglichkeit, ausführliche, gültige, verlässliche Bewertung ohne [Software-Hilfsmittel bereitzustellen]"~\cite{kay-1994}\footnote{
    Übersetzt aus dem Englischen: "impossibility of thorough, valid, reliable assessment without [software tools]".
}, festgestellt.
Eine gängige Vorgehensweise ist es, die abgegebenen Programme mit bestimmten Eingaben auszuführen und die entstehenden Ausgaben zu überprüfen.
In dieser Arbeit wird stattdessen ein Ansatz der textorientierten Suche vorgeschlagen.
Dieser wurde bisher hauptsächlich für das Finden von Plagiaten eingesetzt~\cite{aiken-2002,prechelt-2003}.

\section{Forschungsfragen}\label{sec:research-questions}

Neben der Implementierung von Software sollen in dieser Arbeit auch einige Grundlagen der automatischen Bewertung erforscht werden.
Die folgenden vier Forschungsfragen sollen als Leitfaden für diese Arbeit dienen.

\subsection[\acs{rq}1]{\ac{rq}1: Kann die Bewertung von Programmieraufgaben sinnvoll automatisiert werden?}\label{subsec:rq1-useful-automation}

Wie eingangs erwähnt, bietet sich die Automatisierung der Bewertung von Programmieraufgaben bei vielen Studierenden mit ähnlichen Lösungen an.
Dies setzt voraus, dass eine einheitliche Aufgabenstellung vorhanden ist, wie es bei Hausaufgaben in Lehrveranstaltungen, nicht aber bei Projekten der Fall ist.
Mit dieser Frage soll erforscht werden, ob und welche Arten von Aufgaben besonders für die automatische Bewertung geeignet sind.

\subsection[\acs{rq}2]{\ac{rq}2: Welcher Mehraufwand wird für die automatisierte Bewertung benötigt?}\label{subsec:rq2-additional-effort}

Gemäß der Zielsetzung ist es naheliegend, dass bei einem optionalen Werkzeug grundsätzlich kein Mehraufwand erforderlich ist.
Um jedoch die möglichen Vorteile zu nutzen, müssen möglicherweise Schulungen, Einrichtung und weitere Schritte durch die Bewertenden erfolgen.
Weiterhin könnte dies umfassen, dass Tests für die Programmieraufgaben geschrieben oder Musterlösungen bereitgestellt werden müssen.
Diese Frage soll untersuchen, wie umfangreich diese weiteren Aufwände sind und wie sie in Relation zum Nutzen der Werkzeuge stehen.

\subsection[\acs{rq}3]{\ac{rq}3: Welche Effektivität und Effizienz kann von der Automatisierung erwartet werden?}\label{subsec:rq3-effectivity-efficiency}

In dieser Arbeit beschreibt der Begriff Effektivität den Anteil der Bewertungen von Teilaufgaben, die automatisch angelegt wurden, \ac{dh} in Relation zur Anzahl der manuellen Bewertungen.
Je höher dieser prozentuale Wert ausfällt, desto weniger Arbeit ist seitens der Bewertenden notwendig gewesen.
Die Effizienz hingegen bezeichnet die Zeit, die dabei eingespart wurde, genauer das Verhältnis von automatisiert eingesparter Arbeitszeit zu manueller Bewertungszeit.
Es ist voraussehbar, dass diese Werte abhängig von der Aufgabenstellung unterschiedlich ausfallen können.
Daher soll ergründet werden, welche Arten von Programmieraufgaben positiven oder negativen Einfluss nehmen.
Zudem soll beobachtet werden, ob Unterschiede entstehen, wenn Lehrveranstaltungen aus niedrigen oder höheren Semestern oder Hausaufgaben von Anfang, Mitte oder Ende eines Vorlesungszeitraums betrachtet werden.

\subsection[\acs{rq}4]{\ac{rq}4: Wie können Aufgaben formuliert oder angepasst werden, um die Effektivität und Effizienz zu steigern?}\label{subsec:rq4-improve-effectivity-efficiency}

Aufbauend auf \acs{rq}3 ist es wünschenswert, die Effektivität oder Effizienz zu steigern, um Bewertungsarbeit zu reduzieren.
Diese Frage soll erforschen, wie dies durch Anpassung der Aufgabenstellung bezweckt werden kann.
Dabei sollen Anpassungen möglichst nicht den Lehrinhalt verfälschen oder Mehraufwand erzeugen.

\section{Aufbau der Arbeit}\label{sec:structure}

Nachfolgend wird zunächst beschrieben, welche Konzepte dieser Arbeit zugrundeliegen.
Kapitel~\ref{ch:basics} enthält dafür eine genaue Beschreibung des Anwendungsfalls und der verwendeten Technologien, auf denen die Software basiert.
Daraufhin beschreibt Kapitel~\ref{ch:implementation} im Detail die konstruierten Werkzeuge und deren Funktionsweise aus technischer Sicht.
In Kapitel~\ref{ch:evaluation} wird dargelegt, wie die Software anhand von Benutzungstests über einen längeren Zeitraum erprobt und iterativ verbessert wurde.
Ebenfalls werden dort erste Daten gesammelt, die für die Beantwortung der \acp{rq} notwendig sind.
Die Auswertung und Beantwortung der Fragen erfolgt schließlich in Kapitel~\ref{ch:results}.
Das Fazit in Kapitel~\ref{ch:conclusion} gibt einen abschließenden Rückblick auf die Erkenntnisse der Arbeit.
Zuletzt werden in Kapitel~\ref{ch:future-work} einige zukünftige Erweiterungsmöglichkeiten vorgestellt.
