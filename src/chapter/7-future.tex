\chapter{Ausblick}\label{ch:future-work}

\todo{
    IntelliJ Plugin.
    Platzhalter in Code Search Snippets.
    Anbindung anderer Abgabesysteme (Moodle?).
}

\section{fulib.org Projects}\label{sec:future-projects}

\section{Umbenennung von Bezeichnern und andere Refactorings}\label{sec:renaming-and-refactoring}

\todom{
    Hier muss alles neu geschrieben werden, war nur ein kurzer Info Dump der dazu diente, die aktuelle vorhandene
    Idee der Variablenumbenennung zu verewigen, welche zuvor an einem unerfolgreichen Abend zusammengehackt wurde,
    sich aber als fruchtlos herausstellte. Bis ich wirklich hier ankomme ist das alles lange her, deshalb steht hier
    jetzt schon mal ein bisschen was zum Gedankengang.
}

Umbenennung von Variablen bei in sich geschlossenen Codeschnipseln möglich.
Beispiel:

\begin{minted}{java}
public static void main(String[] args) {
    int i = 6;
    int j = 7;
    int answer = i * j;
    String greeting = "Hello";
    System.out.println(greeting);
    System.out.println("Answer: " + answer);
}
\end{minted}

Ersetzung von langen Bezeichnern durch Typ-Anfangsbuchstabe + laufende Nummer.
Beibehaltung von Offsets.

\begin{minted}{java}
public static void main(String[] s0  ) {
    int i = 6;
    int j = 7;
    int i0     = i * j;
    String s1       = "Hello";
    System.out.println(s1      );
    System.out.println("Answer: " + i0    );
}
\end{minted}

Problem: Variablendeklaration nicht zwangsweise in der Suchanfrage:

\begin{minted}{java}
    System.out.println(greeting);
    System.out.println("Answer: " + answer);
\end{minted}

greeting und answer können von außerhalb stammen;
Typ ist nicht bekannt -
Umbenennung kann nicht reproduziert werden.

Typ-Problem kann gelöst werden durch Benennung wie var0, var1, \ldots.
Nummerierung dann weiterhin Problem.

Alternative: Alle Bezeichner gleich behandeln - dann aber Verlust von Semantikinformationen.
