\chapter{Ausblick}\label{ch:future-work}

Eine Vielzahl von Erweiterungen und neuen Komponenten wurde für diese Arbeit zunächst in Betracht gezogen, aber später verworfen.
Die Ideen, deren Nutzen und die Gründe, sie noch nicht umzusetzen, werden in diesem Kapitel diskutiert.
Dabei wird zwischen drei großen Gebieten unterschieden:
Integration weiterer Lernsysteme, Portierung von fulibFeedback auf andere \acp{ide} und Verbesserungen von Code Search.

\section{Integration weiterer Lernsysteme}

Bisher unterstützt fulib.org den automatischen Import von Lösungen mit GitHub Classroom und das Hochladen von Archiven.
Dies deckt die Anwendungsfälle von den Veranstaltungen aus Kapitel~\ref{ch:evaluation} ab, doch verwenden nicht alle Lehrenden diese Plattform oder Formate.
Bei Abweichungen kann es zu Mehraufwand kommen, beispielsweise wenn abgegebene Dateien zunächst in Archive verpackt werden müssen.
Folglich kann es sinnvoll sein, auch andere Lernsysteme zu integrieren.

Ein Beispiel für ein solches Lernsystem ist Moodle\footnote{\url{https://moodle.org/}}.
Dort können Lehrende ebenfalls Assignments erstellen\footnote{\url{https://docs.moodle.org/311/en/Assignment_activity}}, für die Studierende ihre Lösungen hochladen können.
Bewertende können die Abgaben einsehen und Punkte mit Feedback vergeben.
Die Moodle-\ac{api}\footnote{\url{https://docs.moodle.org/dev/Core_APIs}} bietet Schnittstellen um maschinell die Lösungen auszulesen und Bewertungen zu hinterlegen.

\section{fulibFeedback für andere \acsp{ide}}\label{sec:other-ides}

In Abschnitt~\ref{subsec:language-server-protocol} wurde bereits dargelegt, dass das \ac{lsp} für Wiederverwendbarkeit ausgelegt ist.
Es liegt daher nahe, dass die \ac{vsc}-Erweiterung fulibFeedback auch für andere \acp{ide} mit \ac{lsp}-Unterstützung portiert werden kann.
Konkret wurde in Betracht gezogen, dies für IntelliJ umzusetzen.
Diese Entwicklungsumgebung wurde in den evaluierten Veranstaltungen bereits eingesetzt.
Das Vorhaben wurde jedoch verworfen, da in \ac{pm} die Vorgabe von \ac{vsc} bestand und diese \ac{ide} auch für die Bewertung verwendet wurde.
Darüber hinaus ist die Unterstützung des \ac{lsp} seitens IntelliJ nicht optimal, wie in Abschnitt~\ref{subsec:visual-studio-code} erwähnt wurde.

Eine weitere Möglichkeit ist das Projects-Modul von fulib.org, das in Abschnitt~\ref{subsec:fulib.org} beschrieben wurde.
Die vereinfachte Online-\ac{ide} könnte ebenfalls die Funktionen von fulibFeedback anbieten und dabei besser mit fulib.org integriert sein, als es mit einem \ac{ide}-Plugin möglich wäre.
Beispielsweise könnte mit einem Button automatisch ein Projekt angelegt und geöffnet werden.
Dabei entfallen mögliche Probleme bei der Einrichtung, beispielsweise die Installation von \acp{sdk} oder Spracherweiterungen.
Eine spezialisierte Oberfläche könnte darüber hinaus eine bessere Abstimmung von Quellcode und Bewertung erzielen.
Dieses Vorhaben wurde für diese Arbeit verworfen, da in der Veranstaltung \ac{pm} zunehmend Anwendungen und Oberflächen mit JavaFX implementiert wurden, die im Browserkontext von fulib.org nicht trivial ausführbar sind.

\section{Verbesserungen von Code Search}\label{sec:code-search-improvements}

Die intensive Verwendung von Code Search hat einige Verbesserungsvorschläge ergeben, die nicht umgesetzt wurden und im Folgenden kurz behandelt werden.
Sie orientieren sich hauptsächlich an den Ursachen für Quellcode-Varianten, die bereits in Abschnitt~\ref{subsec:pm-adaptations} dargelegt wurden.

Während die Einrückung und Formatierung bereits ignoriert werden, sind Kommentare dennoch für die Suche relevant.
Dies kann Nachteile haben, wenn sie von Studierenden verfasst wurden und Unterschiede hervorrufen.
Andererseits wurden vorgegebene Kommentare des Öfteren als Kontext für die Suche verwendet, um ungültige Ergebnisse zu vermeiden.
Es wäre denkbar, Kommentare bei der Suche optional zu ignorieren, beispielsweise mit einem zusätzlichen Haken im Bewertungs-Fenster.

Die unterschiedliche Benennung von Bezeichnern könnte umgangen werden, indem vor der Suche eine automatische Umbenennung stattfindet.
Denkbar wäre eine laufende Nummerierung wie \code{var1}, \code{var2}, \ldots.
Eine solche Umbenennung würde konsistente Ergebnisse liefern, sofern die Variablen von jedem Studierenden in der gleichen Reihenfolge deklariert werden.
Bei einfachen Methoden mit nur einer Variable wäre dies trivial erfüllt, bei komplexeren jedoch selten nützlich.
Diese Art der Umbenennung ist außerdem nur möglich, wenn ein in sich geschlossener Codeabschnitt, beispielsweise eine ganze Methode, ausgewählt wird und Klassennamen und -variablen vorgegeben sind.
Wenn hingegen keine Deklaration im ausgewählten Code enthalten ist, kann keine sinnvolle Umbenennung durchgeführt werden, da Variablen syntaktisch nicht von anderen Bezeichnern unterschieden werden können.
Eine Alternative wäre die identische Behandlung aller Bezeichner, jedoch würde dies nicht die Semantik des Codes erhalten und könnte Ergebnisse liefern, die unterschiedliche Funktionalität haben.

Das Problem der Strukturierung ist nicht ohne sprachspezifische Regeln zum Refactoring umsetzbar.
Da mit den in dieser Arbeit vorgestellten Werkzeugen die Sprachunabhängigkeit angestrebt wurde, war die Lösung dieses Problems nicht geplant.

Bezüglich der Beispielwerte wäre es denkbar, bei der Auswahl von Codeabschnitten das Festlegen von Lücken zu erlauben.
Diese könnten von der Suche als Wildcard behandelt werden und somit Quellcode in Lösungen mit anderen Werten ergeben.
Das Konzept der Wildcards könnte auch auf größere Blöcke von irrelevantem Code ausgeweitet werden.
Für diese Arbeit wurden sie nicht in Betracht gezogen, da der zusätzliche Implementierungsaufwand in Oberfläche und Suche nicht die geringfügigen Einbußen bei der Code Search-Effektivität gerechtfertigt haben.
