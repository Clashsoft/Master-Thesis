\chapter{Auswertung}\label{ch:results}

Die Evaluation hat eine Vielzahl von Erkenntnissen und Daten geliefert, die nun ausgewertet werden.
Dabei wird zunächst auf die Zielsetzung aus Abschnitt~\ref{sec:goals} Bezug genommen.
Weiterhin werden die Forschungsergebnisse präsentiert, die Antworten auf die Forschungsfragen aus Abschnitt~\ref{sec:research-questions} liefern.

\section{Zielerfüllung}\label{sec:goals-reached}

Zu Beginn wurden einige Anforderungen und Rahmenbedingungen gestellt, welche die Software erfüllen sollte, die für diese Arbeit entwickelt wurde.
Zunächst sollte sie eine unterstützende Funktion bei der Bewertung von Abgaben erfüllen.
Dies wurde ermöglicht, indem Lösungen automatisch importiert werden, die auf Knopfdruck im Editor geöffnet werden können, und Bewertungen angelegt werden können.
Vorherige Abläufe wie das manuelle Clonen von GitHub, Zusammenstellen von Feedback und Berechnung von Punkten wurden dadurch ersetzt.
Es ist möglich, Aufgaben bis auf einzelne Punkte zu definieren und dadurch detaillierte Statistiken über Erfolge bei kleinen Schritten zu ermitteln.
Mehrere Abgaben können durch Code Search gleichzeitig bewertet werden, wodurch wertvolle Bewertungszeit gespart wird.

Die Verwendung der Werkzeuge ist gänzlich optional, sowohl für Studierende als auch Bewertende.
Bestehende Abgabemechanismen wie GitHub Classroom bleiben erhalten und auch Feedback kann weiterhin mit Issues oder anderen Mitteln eingereicht werden.
Die Software unterstützt ferner den Import von Lösungen beliebiger Abgabesysteme in Form von Dateien sowie das Versenden von Feedback via Email oder Text.
Damit wird die Integration beliebiger Plattformen ermöglicht.
Bewertende, welche die Werkzeuge nicht verwenden möchten, können weiterhin auf bewährte Methoden zurückgreifen.
Sie verzichten dabei jedoch auf die Vorteile von Code Search und sonstiger gesteigerter Produktivität.

\section{Forschungsergebnisse}\label{sec:research-results}

\todo{
    Bezugnahme/Wiederholung von~\ref{sec:research-questions}.
    Auswertung der Statistiken der Code Search-Effektivität in Bezug auf Aufgabentypen, Studentenanzahl und Modul.
}
